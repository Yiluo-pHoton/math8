\documentclass[12pt,letterpaper]{hmcpset}
\usepackage[margin=1in]{geometry}
\usepackage{graphicx}
\usepackage{amsmath,mathpazo}
\usepackage{fancyhdr}
\usepackage{physics}

% info for header block in upper right hand corner


\pagestyle{fancy}
\rhead{ \textsc{Yiluo Li} \ (\textsc{Page} \thepage)  \\ Math 8 \ HW6 \\ 13 November 2017}
\begin{document}

\problemlist{Complex Numbers}
%%%%%%%%%%%%%%%%%%%%%%%%%%%%%%%%%%%%%%%%%%%%%%%%%%%%%%%%%%%%%%% 3
\begin{problem}\textbf{Problem 3}
\newline(a) Find the real and imaginary parts of $(\sqrt{3} - i)^{10}$ and $(\sqrt{3} - i)^{-7}$. For
  which values of n is $(\sqrt{3} - i)^{n}$ real? \newline (b) What is $\sqrt{i}$ ?
\end{problem}

\begin{solution}
\textbf{(a)}
\newline\newline \textbf{PART ONE:} We can write $\sqrt{3} - i$ in the polar form: $2(cos\frac{\pi}{6}-isin\frac{\pi}{6})$. Then by De Moivre's Theorem, we can write $(\sqrt{3} - i)^{10}$ as $2^{10}(cos\frac{5\pi}{3}-isin\frac{5\pi}{3})$ and simplifies to $2^{9}+2^{9}\sqrt{3}i$ in which the real part is $2^{9}$ and the imaginary part is $2^{9}\sqrt{3}i$.
\newline\newline \textbf{PART TWO:} By De Moivre's Theorem, we can write $(\sqrt{3} - i)^{-7}$ as $2^{-7}(cos\frac{-7\pi}{6}-isin\frac{-7\pi}{6})$ and simplifies to $-2^{-8}-2^{-8}\sqrt{3}i$ in which the real part is $2^{-8}$ and the imaginary part is $2^{-8}\sqrt{3}i$.
\newline\newline \textbf{PART THREE:} To rewrite$(\sqrt{3} - i)^{n}$ we have $2^{n}(cos\frac{n\pi}{6} + isin\frac{n\pi}{6})$. If we want to the imaginary part to be gone, $sin\frac{n\pi}{6}$ must be 0, and subsequently $\frac{n\pi}{6}$ must be a multiple of $\pi$, Therefore, n must be a multiple of 6. 
\end{solution}

\begin{solution}
\textbf{(b)} To rewrite the statement of the questions, we get to solve the following equation, in which the second root of unity are the solution to this question:
\begin{align*}
z^{2}	&= i		\\
z^{2}	&= e^{\frac{\pi}{2}i}
\end{align*}
Apparently $\alpha = e^{\frac{\pi}{4}i}$ is a second root of unity, and we have $(\alpha w)^{2} = \alpha ^{2} = i$, so $\alpha w$ is also a second root of unity. In this case, $w=e^{\frac{2k\pi}{n}i}=e^{k\pi i}$ in which k = 0, 1. Now we plug in value of w and find the second root of unity $\alpha w = e^{\frac{\pi}{4}i}$ or $e^{\frac{5\pi}{4}i}$.
\end{solution}

%%%%%%%%%%%%%%%%%%%%%%%%%%%%%%%%%%%%%%%%%%%%%%%%%%%%%%%%%%%%%%% 5
\begin{problem}\textbf{Problem 5}
Let z be a non-zero complex number. Prove that the three cube roots of z are the corners of an equilateral triangle in the Argand diagram.
\end{problem}

\begin{solution}
We can write the problem as following:
\begin{align*}
x^{3}	&= z	\\
x^{3}	&= re^{i\theta}
\end{align*}
In which we let $re^{i\theta}$ denote an arbitrary complex number. Now we can solve this. First find $alpha$, which is an apparent third root of unity to this equation, and then we will find w, which will serve to encapsulate all third root of unity:
\begin{align*}
\alpha &= r^{\frac{1}{3}}e^{\frac{1}{3}i\theta}	\\
w	&= e^{\frac{2k\pi}{n}i}	= e^{\frac{2}{3}\pi i}	\\
\alpha w^{0}	&= r^{\frac{1}{3}} e^{\frac{1}{3}\theta i} = r^{\frac{1}{3}}(cos\frac{1}{3}\theta + isin\frac{1}{3}\theta)	\\
\alpha w^{1}	&= r^{\frac{1}{3}} e^{(\frac{1}{3}\theta + \frac{2}{3} \pi)i} = r^{\frac{1}{3}}(cos(\frac{1}{3}\theta+\frac{2}{3}\pi) + isin(\frac{1}{3}\theta+\frac{2}{3}\pi)	\\
\alpha w^{2}	&= r^{\frac{1}{3}} e^{(\frac{1}{3}\theta + \frac{4}{3} \pi)i} = r^{\frac{1}{3}}(cos(\frac{1}{3}\theta+\frac{4}{3}\pi) + isin(\frac{1}{3}\theta+\frac{4}{3}\pi))	\\
\end{align*} Now we have all three points. Do a distance check between all pairs of two. And for sanity's sake, we will compare the square of their distances:
\begin{align*}
d^2_{\alpha w^{0}-\alpha w^{2}}	&= r^{\frac{2}{3}} (-2cos\frac{1}{3}\theta cos(\frac{1}{3}\theta+\frac{4}{3}\pi))	 = -2r^{\frac{2}{3}}cos(\frac{4\pi}{3})\\
d^2_{\alpha w^{1}-\alpha w^{2}}	&= r^{\frac{2}{3}}(-2cos(\frac{1}{3}\theta+\frac{2}{3}\pi) cos(\frac{1}{3}\theta+\frac{4}{3}\pi)) = -2r^{\frac{2}{3}}cos(-\frac{2\pi}{3})\\
d^2_{\alpha w^{0}-\alpha w^{1}}	&= r^{\frac{2}{3}}(-2cos\frac{1}{3}\theta cos(\frac{1}{3}\theta+\frac{2}{3}\pi)) = -2r^{\frac{2}{3}}cos(\frac{2\pi}{3})\\
\end{align*} Since we know that $cos(\frac{2\pi}{3})=cos(\frac{-2\pi}{3})=cos(\frac{4\pi}{3})$, we verify that the distances between three points are the same. Therefore, the three cube roots of z are the corners of an equilateral triangle in the Argand diagram.
\end{solution}

%%%%%%%%%%%%%%%%%%%%%%%%%%%%%%%%%%%%%%%%%%%%%%%%%%%%%%%%%%%%%%% 6
\begin{problem}\textbf{Problem 6}
Express $\frac{1+i}{\sqrt{3}+i}$ in the form $x+iy$, where x,y $\in\mathbb{R}$. By writing each of 1+i and $\sqrt{3}+i$ in polar form, deduce that
\begin{align*}
cos\frac{\pi}{12}=\frac{\sqrt{3}+1}{2\sqrt{2}},\ sin\frac{\pi}{12}=\frac{\sqrt{3}-1}{2\sqrt{2}}
\end{align*}
\end{problem}

\begin{solution}
\textbf{PART ONE:} 
\begin{align*}
\frac{1+i}{\sqrt{3}+i}	&= \frac{(1+i)(\sqrt{3} - i)}{4}	\\
						&= \frac{1+\sqrt{3}}{4} - i\frac{1-\sqrt{3}}{4}
\end{align*}
\textbf{PART TWO:} We express the $\frac{1+i}{\sqrt{3}+i}$ in polar form:
\begin{align*}
1 + i &= \sqrt{2}(cos\frac{\pi}{4} + isin\frac{\pi}{4})	\\
\sqrt{3} + i 	&= 2(cos\frac{\pi}{6} + isin\frac{\pi}{6})	\\
\frac{1+i}{\sqrt{3}+i}	&= \frac{\sqrt{2}(cos\frac{\pi}{4} + isin\frac{\pi}{4})}{2(cos\frac{\pi}{6} + isin\frac{\pi}{6})}	\\
						&=  \frac{\sqrt{2}(cos\frac{\pi}{4} + isin\frac{\pi}{4})(cos\frac{\pi}{6} - isin\frac{\pi}{6})}{2(cos\frac{\pi}{6} + isin\frac{\pi}{6})(cos\frac{\pi}{6} - isin\frac{\pi}{6})}\\
						&= \frac{\sqrt{2}}{2}(cos\frac{\pi}{4}cos\frac{\pi}{6} + sin\frac{\pi}{4}sin\frac{\pi}{6} + i(sin\frac{\pi}{4}cos\frac{\pi}{6} - cos\frac{\pi}{4}sin\frac{\pi}{6}))\\
						&= \frac{\sqrt{2}}{2}(cos(\frac{\pi}{4} - \frac{\pi}{6}) + isin(\frac{\pi}{4}-\frac{\pi}{6}))\\
						&= \frac{\sqrt{2}}{2}cos\frac{\pi}{12} + \frac{\sqrt{2}}{2}isin\frac{\pi}{12}
\end{align*}
Now we equate this to the form we obtained in PART ONE:
\begin{align*}
\frac{\sqrt{2}}{2}cos\frac{\pi}{12} + \frac{\sqrt{2}}{2}isin\frac{\pi}{12}	&= \frac{1+\sqrt{3}}{4} - i\frac{1-\sqrt{3}}{4}	\\
\end{align*}
We can now separately equate the real and imaginary parts:
\begin{align*}
\frac{\sqrt{2}}{2}cos\frac{\pi}{12}	&= \frac{1+\sqrt{3}}{4} 	\\
cos\frac{\pi}{12}	&= \frac{1+\sqrt{3}}{2\sqrt{2}}	\\
\frac{\sqrt{2}}{2}sin\frac{\pi}{12}	&= - \frac{1-\sqrt{3}}{4}	\\
sin\frac{\pi}{12}	&= \frac{\sqrt{3}-1}{2\sqrt{2}}	\\
\end{align*}
\end{solution}

%%%%%%%%%%%%%%%%%%%%%%%%%%%%%%%%%%%%%%%%%%%%%%%%%%%%%%%%%%%%%%% 8
\begin{problem}\textbf{Problem 8}
Find a formula for $cos4\theta$ in terms of $cos\theta$. Hence write down a quartic equation (i.e., an equation of degree 4) that has $cos\frac{\pi}{12}$ as a root. What are the other roots of your equation?
\end{problem}

\begin{solution}
First we express $cos4\theta$ in terms of $cos\theta$, and we denote $cos\theta$ as c, $sin\theta$ as s:
\begin{align*}
(cos\theta + isin\theta)^{4}	&= (c^{2}+2ics-s^{2})^{2}	\\
cos4\theta + isin4\theta		&= c^{4}-6c^{2}s^{2} + 4ic^{3}s - 4ics^{3} + s^{4}	\\
cos4\theta	&= c^{4} - 6c^{2}s^{2} + s^{4}  \\
\frac{1}{2}	&= x^{4} - 6x^{2}(1-x^{2})+(1-x^{2})^{2}	\\
\frac{1}{2}	&= 8x^{4} - 8x^{2} + 1
\end{align*}We know that $cos4\theta$ with $\theta = \pm\frac{\pi}{12} + \frac{2\pi k}{4}$ will yield the same result. So after eliminating the redundant ones, we are left with four distinct solutions: \newline$\theta=\boxed{\frac{\pi}{12}},\ \boxed{-\frac{\pi}{12}}, \frac{\pi}{12} + \frac{2\pi}{4} = \boxed{\frac{7\pi}{12}},\ -\frac{\pi}{12}+\frac{2\pi}{4}=\boxed{\frac{5\pi}{12}}$
\end{solution}


% Add pairs of problems and solutions as needed

\end{document}
